%% Generated by Sphinx.
\def\sphinxdocclass{report}
\documentclass[letterpaper,10pt,english]{sphinxmanual}
\ifdefined\pdfpxdimen
   \let\sphinxpxdimen\pdfpxdimen\else\newdimen\sphinxpxdimen
\fi \sphinxpxdimen=.75bp\relax

\PassOptionsToPackage{warn}{textcomp}
\usepackage[utf8]{inputenc}
\ifdefined\DeclareUnicodeCharacter
% support both utf8 and utf8x syntaxes
  \ifdefined\DeclareUnicodeCharacterAsOptional
    \def\sphinxDUC#1{\DeclareUnicodeCharacter{"#1}}
  \else
    \let\sphinxDUC\DeclareUnicodeCharacter
  \fi
  \sphinxDUC{00A0}{\nobreakspace}
  \sphinxDUC{2500}{\sphinxunichar{2500}}
  \sphinxDUC{2502}{\sphinxunichar{2502}}
  \sphinxDUC{2514}{\sphinxunichar{2514}}
  \sphinxDUC{251C}{\sphinxunichar{251C}}
  \sphinxDUC{2572}{\textbackslash}
\fi
\usepackage{cmap}
\usepackage[T1]{fontenc}
\usepackage{amsmath,amssymb,amstext}
\usepackage{babel}



\usepackage{times}
\expandafter\ifx\csname T@LGR\endcsname\relax
\else
% LGR was declared as font encoding
  \substitutefont{LGR}{\rmdefault}{cmr}
  \substitutefont{LGR}{\sfdefault}{cmss}
  \substitutefont{LGR}{\ttdefault}{cmtt}
\fi
\expandafter\ifx\csname T@X2\endcsname\relax
  \expandafter\ifx\csname T@T2A\endcsname\relax
  \else
  % T2A was declared as font encoding
    \substitutefont{T2A}{\rmdefault}{cmr}
    \substitutefont{T2A}{\sfdefault}{cmss}
    \substitutefont{T2A}{\ttdefault}{cmtt}
  \fi
\else
% X2 was declared as font encoding
  \substitutefont{X2}{\rmdefault}{cmr}
  \substitutefont{X2}{\sfdefault}{cmss}
  \substitutefont{X2}{\ttdefault}{cmtt}
\fi


\usepackage[Bjarne]{fncychap}
\usepackage{sphinx}

\fvset{fontsize=\small}
\usepackage{geometry}

% Include hyperref last.
\usepackage{hyperref}
% Fix anchor placement for figures with captions.
\usepackage{hypcap}% it must be loaded after hyperref.
% Set up styles of URL: it should be placed after hyperref.
\urlstyle{same}
\addto\captionsenglish{\renewcommand{\contentsname}{Contents:}}

\usepackage{sphinxmessages}
\setcounter{tocdepth}{1}



\title{gps2net}
\date{Dec 03, 2019}
\release{1.1.2020}
\author{Joachim Baumann}
\newcommand{\sphinxlogo}{\vbox{}}
\renewcommand{\releasename}{Release}
\makeindex
\begin{document}

\pagestyle{empty}
\sphinxmaketitle
\pagestyle{plain}
\sphinxtableofcontents
\pagestyle{normal}
\phantomsection\label{\detokenize{index::doc}}

\index{gps2net (module)@\spxentry{gps2net}\spxextra{module}}\index{distFrom() (in module gps2net)@\spxentry{distFrom()}\spxextra{in module gps2net}}

\begin{fulllineitems}
\phantomsection\label{\detokenize{index:gps2net.distFrom}}\pysiglinewithargsret{\sphinxcode{\sphinxupquote{gps2net.}}\sphinxbfcode{\sphinxupquote{distFrom}}}{\emph{lng1}, \emph{lat1}, \emph{lng2}, \emph{lat2}}{}
Returns the distance between two points in meters.
\begin{quote}\begin{description}
\item[{Parameters}] \leavevmode\begin{description}
\item[{\sphinxstylestrong{lng1}}] \leavevmode{[}float{]}
This fff float is the longitude of the source.

\item[{\sphinxstylestrong{lat1}}] \leavevmode{[}float{]}
This float is the latitude of the source.

\item[{\sphinxstylestrong{lng2}}] \leavevmode{[}float{]}
This float is the longitude of the target.

\item[{\sphinxstylestrong{lat2}}] \leavevmode{[}float{]}
This float is the latitude of the target.

\end{description}

\item[{Returns}] \leavevmode\begin{description}
\item[{\sphinxstylestrong{dist}}] \leavevmode{[}int{]}
The distance between two points in meters.

\end{description}

\end{description}\end{quote}
\subsubsection*{Notes}

This function calculates the distance between two gps points.
The result is not 100\% correct as the function does not consider the elipsis-like shape of the earth. Instead it just uses an earth radius of 6371000 meters for the calculation.

A more accurate calculation of the distance can be done in QGIS. For more details, please visit: \sphinxurl{http://www.qgistutorials.com/en/docs/calculating\_line\_lengths.html}
However, this function was used to be able to run the code independently of QGIS.
\subsubsection*{Examples}

\begin{sphinxVerbatim}[commandchars=\\\{\}]
\PYG{g+gp}{\PYGZgt{}\PYGZgt{}\PYGZgt{} }\PYG{n}{x}\PYG{o}{=}\PYG{l+m+mi}{12}
\PYG{g+gp}{\PYGZgt{}\PYGZgt{}\PYGZgt{} }\PYG{n}{x}
\PYG{g+go}{12}
\PYG{g+gp}{\PYGZgt{}\PYGZgt{}\PYGZgt{} }\PYG{p}{(}\PYG{l+m+mi}{5}\PYG{o}{\PYGZlt{}}\PYG{l+m+mi}{10}\PYG{p}{)}
\PYG{g+go}{True}
\PYG{g+gp}{\PYGZgt{}\PYGZgt{}\PYGZgt{} }\PYG{n}{myDist} \PYG{o}{=} \PYG{n}{distFrom}\PYG{p}{(}\PYG{o}{\PYGZhy{}}\PYG{l+m+mf}{122.115}\PYG{p}{,} \PYG{l+m+mf}{37.115}\PYG{p}{,} \PYG{o}{\PYGZhy{}}\PYG{l+m+mf}{122.111}\PYG{p}{,} \PYG{l+m+mf}{37.111}\PYG{p}{)}
\PYG{g+gp}{\PYGZgt{}\PYGZgt{}\PYGZgt{} }\PYG{n}{myDist}
\PYG{g+go}{568.8872918546489}
\PYG{g+gp}{\PYGZgt{}\PYGZgt{}\PYGZgt{} }\PYG{n}{distFrom}\PYG{p}{(}\PYG{o}{\PYGZhy{}}\PYG{l+m+mf}{122.115}\PYG{p}{,} \PYG{l+m+mf}{37.115}\PYG{p}{,} \PYG{o}{\PYGZhy{}}\PYG{l+m+mf}{122.111}\PYG{p}{,} \PYG{l+m+mf}{37.111}\PYG{p}{)} 
\PYG{g+go}{5}
\end{sphinxVerbatim}

\begin{sphinxVerbatim}[commandchars=\\\{\}]
\PYG{g+gp}{\PYGZgt{}\PYGZgt{}\PYGZgt{} }\PYG{n+nb}{print}\PYG{p}{(}\PYG{n}{myDistance}\PYG{p}{)} \PYG{o}{/}\PYG{o}{/} \PYG{n}{prints} \PYG{l+m+mf}{568.8872918} \PYG{o}{.}\PYG{o}{.}\PYG{o}{.}
\end{sphinxVerbatim}

\end{fulllineitems}

\index{getFilename() (in module gps2net)@\spxentry{getFilename()}\spxextra{in module gps2net}}

\begin{fulllineitems}
\phantomsection\label{\detokenize{index:gps2net.getFilename}}\pysiglinewithargsret{\sphinxcode{\sphinxupquote{gps2net.}}\sphinxbfcode{\sphinxupquote{getFilename}}}{\emph{path}}{}
Returns the name of a file from a specific path (excluding directories and extension).
\begin{quote}\begin{description}
\item[{Parameters}] \leavevmode\begin{description}
\item[{\sphinxstylestrong{path}}] \leavevmode{[}{[}string{]}{]}
The path of a specific data file.

\end{description}

\item[{Returns}] \leavevmode\begin{description}
\item[{string}] \leavevmode
The name of the file. Namely, the last part of the path (excluding the extgension).

\end{description}

\end{description}\end{quote}
\subsubsection*{Examples}

\begin{sphinxVerbatim}[commandchars=\\\{\}]
\PYG{g+gp}{\PYGZgt{}\PYGZgt{}\PYGZgt{} }\PYG{n}{myPath} \PYG{o}{=} \PYG{l+s+s1}{\PYGZsq{}}\PYG{l+s+s1}{dir1/dir2/MyFilename.txt}\PYG{l+s+s1}{\PYGZsq{}}
\PYG{g+gp}{\PYGZgt{}\PYGZgt{}\PYGZgt{} }\PYG{n}{filename} \PYG{o}{=} \PYG{n}{getFilename}\PYG{p}{(}\PYG{n}{myPath}\PYG{p}{)}
\PYG{g+gp}{\PYGZgt{}\PYGZgt{}\PYGZgt{} }\PYG{n}{filename}
\PYG{g+go}{\PYGZsq{}MyFilename\PYGZsq{}}
\end{sphinxVerbatim}

\end{fulllineitems}

\index{printProgressBar() (in module gps2net)@\spxentry{printProgressBar()}\spxextra{in module gps2net}}

\begin{fulllineitems}
\phantomsection\label{\detokenize{index:gps2net.printProgressBar}}\pysiglinewithargsret{\sphinxcode{\sphinxupquote{gps2net.}}\sphinxbfcode{\sphinxupquote{printProgressBar}}}{\emph{iteration}, \emph{total}, \emph{prefix=''}, \emph{suffix=''}, \emph{decimals=1}, \emph{length=100}, \emph{fill='█'}, \emph{printEnd='r'}}{}
Call in a loop to create terminal progress bar.
\begin{quote}\begin{description}
\item[{Parameters}] \leavevmode\begin{description}
\item[{\sphinxstylestrong{iteration   - Required}}] \leavevmode{[}current iteration (Int){]}
total       - Required  : total iterations (Int)
prefix      - Optional  : prefix string (Str)
suffix      - Optional  : suffix string (Str)
decimals    - Optional  : positive number of decimals in percent complete (Int)
length      - Optional  : character length of bar (Int)
fill        - Optional  : bar fill character (Str)
printEnd    - Optional  : end character (e.g. “

\item[{\sphinxstylestrong{“, “}}] \leavevmode
\item[{\sphinxstylestrong{“) (Str)}}] \leavevmode
\end{description}

\end{description}\end{quote}

\end{fulllineitems}



\chapter{Indices and tables}
\label{\detokenize{index:indices-and-tables}}\begin{itemize}
\item {} 
\DUrole{xref,std,std-ref}{genindex}

\item {} 
\DUrole{xref,std,std-ref}{modindex}

\item {} 
\DUrole{xref,std,std-ref}{search}

\end{itemize}


\renewcommand{\indexname}{Python Module Index}
\begin{sphinxtheindex}
\let\bigletter\sphinxstyleindexlettergroup
\bigletter{g}
\item\relax\sphinxstyleindexentry{gps2net}\sphinxstyleindexpageref{index:\detokenize{module-gps2net}}
\end{sphinxtheindex}

\renewcommand{\indexname}{Index}
\printindex
\end{document}